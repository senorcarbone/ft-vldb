%!TEX root = main.tex

\section{Conclusion and Future Work}
\label{sec:conclusion}

We presented Apache Flink's core mechanisms for managing persistent, large-scale pipelines with large application state in production. Flink is a flexible, reconfigurable distributed system which runs continuous, analytical, data-centric computation offering strong state consistency guarantees. A distinct snapshotting mechanism acquires a global view of the system periodically or upon demand which allows for coarse grained rollback recovery in a asynchronous, transparent and efficient manner. Snapshots allow for fundamentally practical reconfiguration usages, ranging from partial failure recovery to application versioning, debugging and general state management. Managed state in Apache Flink can be declared through state interfaces (both values and collections) that abstract runtime state management concerns from the programmer. Finally, we demonstrated the low execution overhead of Flink's snapshots in large-scale production deployments.

\vspace{-1mm}
\para{Future Work:} Our main future focus on Apache Flink is to further improve its state management capabilities with asynchronous main memory state and incremental snapshots, enabling automated system estimation of throughput and reconfiguration latency trade-offs. Furthermore, we are planning to include the capability of auto-scaling pipelines according to runtime requirements without user circumvention. \gyula{shuould we write 1-2 more sentences on how this will be done? None of them are super complicated} Finally, we want to support deeper stateful analytics through efficient structured iterative processing, one of the biggest challenges in continuous processing today. 

\vspace{-1mm}
\para{Acknowledgments:} We would like to thank all reviewers who gave us valuable feedback and the whole community of Flink's committers and contributors who offered ideas and implementations to the system. 