%!TEX root = main.tex

\section{Conclusion and Future Work}
\label{sec:conclusion}

We presented Apache Flink's core mechanisms for managing peristent, large-scale pipelines with large application state in production. Flink is a flexible, reconfigurable distributed system which runs continuous, analytical, data-centric computation offering strong state consistency guarantees. A distinct snapshotting mechanism aquires a global view of the system periodically or upon demand which allows for coarse grained rollback recovery in a asynchronous, transparent and efficient manner. Snapshots allow for fundamentally practical reconfiguration usages, ranging from partial failure recovery to application versioning, debugging and general state management. Managed state in Apache Flink can be trivially declared through special collections that abstract runtime state management concerns from the programmer. We demonstrated Flink's practical impact with large-scale deployment measurements with terabytes of state, continuously updated by transformations triggered by millions of continuous streams of input data.

Our main future focus on Apache Flink is to improve further its state management capabilities with asynchronous main memory state and incremental snapshots, enabling automated system estimation of throughput and reconfiguration latency trade-offs. Furthermore, we are planning to include the capability of auto-scaling pipelines according to runtime requirements without user circumvention. Finally, we want to support deeper stateful analytics through efficient structured iterative processing, one of the biggest challenges in continuous processing today.